\documentclass{article}
\usepackage{amsmath}
\usepackage{amssymb}
\usepackage{graphicx}

\begin{document}
%%%%%%%%%%%%%%%%%%%%%%%%%%%%%%%%%%%%%%%%%%%%%%%%%%%%%%%%%%%%%%%%%%%%%
\noindent
\rule{\textwidth}{1pt}
\begin{center}
{\bf [CS309] Foundations of Cryptography and Security Networks}
\end{center}
Instructor: Dr. Dibyendu Roy \hfill Autumn Semester 2024\\
Notes by: Tanuj Saini (202251141) \hfill Week 7 Lecture
\\
\rule{\textwidth}{1pt}
%%%%%%%%%%%%%%%%%%%%%%%%%%%%%%%%%%%%%%%%%%%%%%%%%%%%%%%%%%%
% Start here

\section{Composition Function}

A compression function is an essential element in cryptographic hash functions. It accepts an input message of length $(m + t)$ and produces an output of fixed length $m$, where $t \geq 1$:

\[ h : \{0, 1\}^{m+t} \rightarrow \{0, 1\}^m \]

The security of the hash function $H$ is contingent upon the security of the compression function $h$.

Given an input $x \in \{0, 1\}$ with a length denoted by $|x|$ (where $|x| \geq m + t + 1$):

Construct $y$ from $x$ using a public function such that $|y| \equiv 0 \mod (t)$.
\begin{itemize}
    \item If $|x| \equiv 0 \mod (t)$, then $y = x$.
    \item If $|x| + d \equiv 0 \mod (t)$, then $y = x \parallel 0^x$.
    \item Otherwise, $y = y_1 \parallel y_2 \parallel \ldots \parallel y_r$ such that $|y| = t$ for $1 \leq i \leq r$.
\end{itemize}

Let $Z_0 = IV$, and compute:

\[ Z_1 = h(Z_0 \parallel y_1) \]

Continuing in this manner, we have:

\[ Z = h(Z_{r-1} \parallel y_r) \]

\section{Merkle--Damg\r{a}rd Construction}

The Merkle-Damg\r{a}rd construction is commonly used in cryptographic hash function design. It breaks the input message into fixed-size blocks and iteratively applies a compression function to produce the final hash value.

\[ h : \{(0, 1)\}^* \rightarrow \{(0, 1)\}^m \]

The compression function:

\[ \text{Compress}: \{(0, 1)\}^{m+t} \rightarrow \{(0, 1)\}^m \]

Let $n = |x|$ and define:

\[ k = \left\lfloor \frac{n}{t} - 1 \right\rfloor \]
\[ d = k(t - 1) - n \]

For $i = 1$ to $k$:
\[ y_i = x_i \]

Construct:
\[ y_m = x_k \parallel O^{(d)} \]
\[ y_{m+1} = \text{binary}(d) \]

Initialization:
\[ z_1 = O^{m+1} \parallel y_1 \]
\[ g_1 = \text{compress}(z_1) \]

For $i = 1$ to $k$:
\[ z_{i+1} = g_i \parallel 1 \parallel y_{i+1} \]
\[ g_{i+1} = \text{compress}(z_{i+1}) \]

Final output:
\[ h(x) = g_{k+1} \]

Return $h(x)$.

\section{Secure Hash Functions (SHA)}

The Secure Hash Algorithm (SHA) was developed by the National Institute of Standards and Technology (NIST) and was first introduced as a federal information processing standard (FIPS 180) in 1993. Over time, revisions were made, with SHA-1 being published in 1995 as FIPS 180-1. Other versions in the SHA family include SHA-224, SHA-256, SHA-384, and SHA-512.

\subsection{SHA-1}

The Secure Hash Algorithm 1 (SHA-1) is a cryptographic hash function designed by the National Security Agency (NSA) and published by the National Institute of Standards and Technology (NIST) as part of the Digital Signature Algorithm (DSA). SHA-1 outputs a fixed-length, 160-bit (20-byte) message digest from an input of arbitrary length, typically represented as a 40-character hexadecimal number.

\subsection{Algorithm}

SHA-1 is constructed using the Merkle-Damg\r{a}rd design and processes input in blocks. The core steps of the SHA-1 algorithm are outlined as follows:

\begin{enumerate}
    \item Message Padding: The input message is padded to ensure its bit length is congruent to 448 mod 512. Padding involves appending a '1' bit, followed by enough '0' bits, and finally adding the 64-bit representation of the message's original length.
    \item Message Parsing: The padded message is split into 512-bit blocks.
    \item Initial Hash Values: SHA-1 starts with five 32-bit registers, denoted $H_0$, $H_1$, $H_2$, $H_3$, $H_4$, initialized as follows:
    \begin{align*}
        H_0 &= \text{0x67452301} \\
        H_1 &= \text{0xEFCDAB89} \\
        H_2 &= \text{0x98BADCFE} \\
        H_3 &= \text{0x10325476} \\
        H_4 &= \text{0xC3D2E1F0}
    \end{align*}
    \item Processing Each Block: For each 512-bit block, the following steps are carried out:
    \begin{enumerate}
        \item The block is divided into sixteen 32-bit words, and an additional 64 words are generated through bitwise operations.
        \item 80 rounds are executed, involving mixing the message schedule words with the hash values, using bitwise operations such as AND, OR, XOR, and rotations.
        \item The result from each block's processing updates the hash values $H_0$ to $H_4$.
    \end{enumerate}
    \item Final Output: After processing all the blocks, the concatenation of $H_0$, $H_1$, $H_2$, $H_3$, $H_4$ produces the final 160-bit hash value.
\end{enumerate}

\subsection{SHA-1 Rounds}

SHA-1 involves 80 rounds of processing for each block, with four constants used based on the round number:

\[
K_t =
\begin{cases}
    \text{0x5A827999} & \text{for } 0 \leq t \leq 19 \\
    \text{0x6ED9EBA1} & \text{for } 20 \leq t \leq 39 \\
    \text{0x8F1BBCDC} & \text{for } 40 \leq t \leq 59 \\
    \text{0xCA62C1D6} & \text{for } 60 \leq t \leq 79
\end{cases}
\]

The core loop of the algorithm applies bitwise logical functions $F_t$ based on the round number $t$, utilizing word additions and rotations.

\section{SHA-256}

SHA-256, part of the SHA-2 family, is a cryptographic hash function that generates a fixed-size 256-bit (32-byte) hash value from an arbitrary-length input. Designed by the NSA and published by NIST, SHA-256 operates similarly to SHA-1 but with notable differences in block size and operations.

\subsection{Steps of SHA-256}

\begin{enumerate}
    \item Message Padding:
    \begin{itemize}
        \item The input message is padded so that its bit length is congruent to 448 mod 512.
        \item Padding involves appending a single '1' bit, followed by '0' bits, and ending with a 64-bit representation of the original message length.
    \end{itemize}
    \item Message Parsing: The padded message is split into 512-bit blocks, which are processed sequentially.
    \item Initial Hash Values: Eight 32-bit registers ($H_0$, $H_1$, $H_2$, $H_3$, $H_4$, $H_5$, $H_6$, $H_7$) are initialized to constants derived from the fractional parts of the square roots of the first eight primes:
    \begin{align*}
        H_0 &= \text{0x6a09e667} \\
        H_1 &= \text{0xbb67ae85} \\
        H_2 &= \text{0x3c6ef372} \\
        H_3 &= \text{0xa54ff53a} \\
        H_4 &= \text{0x510e527f} \\
        H_5 &= \text{0x9b05688c} \\
        H_6 &= \text{0x1f83d9ab} \\
        H_7 &= \text{0x5be0cd19}
    \end{align*}
    \item Processing Each Block:
    \begin{enumerate}
        \item Message Schedule: A sequence of 64 words $W_t$ is generated from each block, with the first 16 directly derived from the block and the remaining 48 produced by:
        \[ W_t = \sigma_1(W_{t-2}) + W_{t-7} + \sigma_0(W_{t-15}) + W_{t-16} \]
        where:
        \begin{align*}
            \sigma_0(x) &= (x \ggg 7) \oplus (x \ggg 18) \oplus (x \ggg 3) \\
            \sigma_1(x) &= (x \ggg 17) \oplus (x \ggg 19) \oplus (x \ggg 10)
        \end{align*}
        \item Compression Function: For each round $t$:
        \begin{align*}
            T_1 &= H + \Sigma_1(E) + Ch(E, F, G) + K_t + W_t \\
            T_2 &= \Sigma_0(A) + Maj(A, B, C)
        \end{align*}
        where:
        \begin{align*}
            \Sigma_0(x) &= (x \ggg 2) \oplus (x \ggg 13) \oplus (x \ggg 22) \\
            \Sigma_1(x) &= (x \ggg 6) \oplus (x \ggg 11) \oplus (x \ggg 25) \\
            Ch(x, y, z) &= (x \land y) \oplus (\neg x \land z) \\
            Maj(x, y, z) &= (x \land y) \oplus (x \land z) \oplus (y \land z)
        \end{align*}
        Variables are updated as follows:
        \[ H = G, G = F, F = E, E = D + T_1, D = C, C = B, B = A, A = T_1 + T_2 \]
    \end{enumerate}
    \item Final Hash: After all blocks are processed, the final hash is obtained by concatenating the updated values:
    \[ \text{Hash} = H_0 \parallel H_1 \parallel H_2 \parallel H_3 \parallel H_4 \parallel H_5 \parallel H_6 \parallel H_7 \]
    The final output is a 256-bit (32-byte) value.
\end{enumerate}

\subsection{Round Constants}

SHA-256 uses 64 constant values $K_t$, one for each round, which are derived from the first 32 bits of the fractional parts of the cube roots of the first 64 primes.

\section{Diffie-Hellman Key Exchange}

The diagram below represents a symmetric key encryption setup:

% Here you would include the image. Since I can't include actual images, I'll leave a placeholder comment.
% \includegraphics[width=\textwidth]{image.png}

The Diffie-Hellman Key Exchange Algorithm introduced a new paradigm in cryptography, commonly referred to as Public Key Cryptography.

$G$ represents a cyclic group defined as $\langle g \rangle$: $(G, *)$

\begin{enumerate}
    \item For any $a, b \in G$, then $a * b \in G$.
    \item There exists an identity element $e \in G$ such that:
    \[ e * a = a * e = a, \forall a \in G. \]
    \item Each $a \in G$ has an inverse $a^{-1} \in G$ such that:
    \[ a * a^{-1} = a^{-1} * a = e \]
    \item $G$ is associative.
\end{enumerate}

\section{RSA}

$\Phi(m)$: the number of integers less than $m$ that are coprime with $m$.

Example: $\Phi(m) = 4 \rightarrow 1, 3, 5, 7$

For a prime number $p$, $\Phi(p) = p - 1$. $p$: prime

For powers of a prime, $\Phi(p^k) = p^k - p^{k-1}$.

Alternatively, $\Phi(p^k) = p^k \left(1 - \frac{1}{p}\right)$.

If $\gcd(a, m) = 1$, then:
$S = x \mod m$
$S$ consists of $\{r_1, r_2, \ldots, r_m\}$.

Multiplying $S$ by $a$ gives: $\{ar_1, ar_2, \ldots, ar_m\}$.

If $ar_i = ar_j$, then $r_i = r_j$. This is possible only if $r_i \neq r_j$.

Since $\gcd(a, m) = 1$, we can express this as:
\[ 1 = a \cdot b + m \cdot s \]

Thus, $\exists$ some $b$ such that $a \cdot b \equiv 1 \mod m$.

If $ar_i \equiv ar_j \mod m$ and $r_i \neq r_j$, we get:
\[ b \cdot ar_i \equiv b \cdot ar_j \mod m. \]

Hence, $r_i \equiv r_j \mod m$, and therefore, $ar_i \neq ar_j \mod m$.

\subsection{Fermat's Little Theorem}

For a prime $p$, if $a$ is an integer not divisible by $p$, then:
\[ a^{p-1} \equiv 1 \pmod{p} \]


\section{Mitigating Man-in-the-Middle Attacks}

\subsection{Endpoint Authentication}
\begin{itemize}
    \item Digital signatures can verify the integrity and authenticity of messages.
    \item Public-key cryptography: Use the private key for signing and the public key for verification.
\end{itemize}

\subsection{Secure Key Exchange Protocols}
\begin{itemize}
    \item Diffie-Hellman key exchange: Enables secure generation of shared secrets.
    \item RSA key exchange: The shared secret is encrypted using the recipient's public key.
\end{itemize}

\section{Euler's Totient Function and Theorem}

\subsection{Definition and Properties}
\begin{itemize}
    \item Definition: $\Phi(m)$ counts the number of integers less than $m$ that are coprime with $m$.
    \item Properties:
    \begin{itemize}
        \item For a prime $p$, $\Phi(p) = p - 1$.
        \item For two distinct primes $p$ and $q$, $\Phi(pq) = (p - 1)(q - 1)$.
        \item For powers of a prime $p$, $\Phi(p^k) = p^k - p^{k-1}$.
    \end{itemize}
\end{itemize}

\subsection{Euler's Theorem}
If $a$ and $m$ are coprime, then $a^{\Phi(m)} \equiv 1 \mod m$. This is essential for RSA and other cryptographic algorithms.

\end{document}