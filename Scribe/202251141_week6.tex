\documentclass{article}
\usepackage{amsmath}
\usepackage{amssymb}
\usepackage{graphicx}

\begin{document}
%%%%%%%%%%%%%%%%%%%%%%%%%%%%%%%%%%%%%%%%%%%%%%%%%%%%%%%%%%%%%%%%%%%%%
\noindent
\rule{\textwidth}{1pt}
\begin{center}
{\bf [CS309] Foundations of Cryptography and Security Networks}
\end{center}
Instructor: Dr. Dibyendu Roy \hfill Autumn Semester 2024\\
Notes by: Tanuj Saini (202251141) \hfill Week 6 Lecture
\\
\rule{\textwidth}{1pt}
%%%%%%%%%%%%%%%%%%%%%%%%%%%%%%%%%%%%%%%%%%%%%%%%%%%%%%%%%%%
% Start here


\section{Stream Cipher}
A stream cipher consists of two components, a pseudo-random bit generator (PRBG) and an encryption and decryption technique. The PRBG generates keystream bits $Z_i$.

\subsection{Encryption}
\[
C_i = m_i \oplus Z_i
\]

\subsection{Decryption}
\[
m_i = C_i \oplus Z_i
\]

Here, XOR is used as the encryption technique, but other efficient computations may be used.

\subsection{Synchronous Stream Cipher}
In a synchronous stream cipher, the keystream is generated independently of the plaintext and ciphertext bits. The functions involved are:

\begin{itemize}
    \item State Update function: $S_{i+1} = f(S_i, K)$
    \item Keystream Generator function: $Z_i = g(S_i, K)$
    \item Ciphertext Generation function: $C_i = h(Z_i, m_i)$
\end{itemize}

Here, $S_0$ is the initial state determined from $K$ and IV. The functions $f$ and $g$ do not take plaintext or ciphertext as inputs in synchronous stream ciphers.

\subsection{Asynchronous or Self-Synchronizing Stream Ciphers}
In asynchronous stream ciphers, the keystream bits are generated as a function of the key and a fixed number of previous ciphertext bits.

\begin{itemize}
    \item State Update Function: $\sigma_{i+1} = f(\sigma, K, IV)$ where $\sigma = (C_{i-t}, C_{i-t+1}, \dots, C_{i-1})$
    \item Keystream Generation Function: $Z_i = g(\sigma_i, K)$
    \item Ciphertext Generation Function: $C_i = h(Z_i, m_i)$
\end{itemize}

The initial state $\sigma_0 = (C_{-t}, C_{-t+1}, \dots, C_{-1})$ is the non-secret initial state.

\section{Linear Feedback Shift Register (LFSR)}
LFSRs are widely used in stream ciphers, for example, in 4G communication. It consists of an $n$-bit register, where the states are denoted by $S$ and the bits by $s$. At each clock cycle, the state updates, and a keystream bit is generated.

\subsection{Example of a 3-bit LFSR}
At clock cycle $t = 0$, the state is:
\[
S_0 = s_2, s_1, s_0
\]

After one clock cycle (right shift for this example), the state becomes:
\[
S_1 = s_1, s_0, s_2
\]
Where the rightmost bit $s_0$ moves out as output, and the feedback bit $s_n$ is calculated as:
\[
s_n = L(s_0, s_1, \dots, s_{n-1})
\]

The function $L$ is a linear function on the previous state:
\[
L(s_0, s_1, \dots, s_{n-1}) = a_0 s_0 \oplus a_1 s_1 \oplus \dots \oplus a_{n-1} s_{n-1}
\]
Where $a_i \in \{0, 1\}$.

\subsection{Linear vs. Affine Functions}
If $a_n = 0$, the function is linear:
\[
L = a_0 s_0 \oplus a_1 s_1 \oplus \dots \oplus a_{n-1} s_{n-1}
\]
If $a_n = 1$, it becomes an affine function.

Example of linear function:
\[
L_1(x, y) = x \oplus y
\]
Example of non-linear function:
\[
L_2(x, y) = 1 \oplus x \oplus y
\]

\section{Period of an LFSR}
The period of an LFSR is defined as the number of clock cycles before the state repeats. The maximum period for an $n$-bit LFSR is $2^n - 1$.

\subsection{Example of a 3-bit LFSR with $L = s_0 \oplus s_2$}
Consider the following example:

At $t=0$:
\[
S_0 = 1, 1, 0
\]
At $t=1$:
\[
S_1 = 0, 1, 1
\]

This LFSR will repeat its states every $2^n - 1 = 7$ clock cycles.

\section{LFSR with Non-Linear Filter Function}
A non-linear filter function takes $l$-bits from the state of the LFSR and applies a non-linear function $f$. This makes the keystream bit $Z_i$ a non-linear function of the state.

The output function is:
\[
Z_i = f(s_0, s_1, \dots, s_{l-1})
\]
This improves security, making it harder to recover the secret key through a linear system of equations.

\section{Hash Functions}
A hash function maps input data to a fixed-size output. The key properties of a good hash function are:

\begin{enumerate}
    \item A small change in input drastically changes the output.
    \item It is computationally infeasible to reverse the function (find the pre-image).
    \item It is infeasible to find two different inputs with the same hash (collision resistance).
\end{enumerate}

\subsection{Example of Pre-Image Finding Problem}
The probability of finding a pre-image can be approximated as:
\[
P[\text{pre-image finding}] \approx \frac{Q}{M}
\]
Where $Q$ is the number of guesses, and $M$ is the output size.

\end{document}
