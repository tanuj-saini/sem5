\documentclass[11pt]{article}
\usepackage[hmargin=1in,vmargin=1in]{geometry}
\usepackage{xcolor}
\usepackage{amsmath,amssymb,amsfonts,url,sectsty,framed,tcolorbox,framed,amsthm, tikz}
\newcommand{\pf}{{\bf Proof: }}
\newtheorem{theorem}{Theorem}
\newtheorem{lemma}{Lemma}
\newtheorem{proposition}{Proposition}
\newtheorem{definition}{Definition}
\newtheorem{remark}{Remark}
\usepackage{amsmath}
\usepackage{array}
\usepackage{algorithm}
\newtheorem{example}{Example}
\usepackage{algorithmic}
\usepackage{titlesec}
\setcounter{secnumdepth}{4}
\begin{document}
%%%%%%%%%%%%%%%%%%%%%%%%%%%%%%%%%%%%%%%%%%%%%%%%%%%%%%%%%%%%%%%%%%%%%
\noindent
\rule{\textwidth}{1pt}
\begin{center}
{\bf [CS309] Introduction to Cryptography and Network Security}
\end{center}
Course Instructor: Dr. Dibyendu Roy \hfill Autumn 2024-2025\\
Scribed by: Tanuj Saini(202251141) \hfill  Week 9,10
\\
\rule{\textwidth}{1pt}

\section{Diffie-Hellman Key Exchange}
The Diffie-Hellman algorithm facilitates secure key exchange using public key cryptography, enabling two users to agree on a shared secret over an insecure channel. Consider a cyclic group $G = \langle g\rangle$ of prime order $P$. Alice selects a private key $a$ $(1 \leq a < P)$ and computes her public key $A = g^a \bmod P$. Similarly, Bob selects $b$ and computes $B = g^b \bmod P$. They exchange $A$ and $B$.

\textbf{Key Computation:}\\
Alice computes: $S = B^a \bmod P = g^{ab}$,\\
Bob computes: $S = A^b \bmod P = g^{ab}$.\\
Shared Secret Key: $S = g^{ab}$, which both parties independently compute.

\textbf{Discrete Logarithm Problem:} Given $g^x \bmod P$, finding $x$ is computationally hard, ensuring the security of the exchange.

\textit{Note:} The security of the Diffie-Hellman algorithm relies on the hardness of the discrete logarithm problem.

\section{Modular Equations}
We'll explore solving systems of linear equations to find $x$ in the form:
\[ a \cdot x \equiv b \bmod m \quad \text{(Eq.1)} \]

To begin, let's express Eq.1 as:
\[ a \cdot x - m \cdot y = b \quad \text{(Eq.2)} \]
where $y$ is an integer. Utilizing Bezout's Identity:
\[ a \cdot x_0 + m \cdot y_0 = \gcd(a, m) \quad \text{(Eq.3)} \]
where $x_0$ and $y_0$ can be determined using the Extended Euclidean Algorithm.

Eq.2 is solvable if and only if $\gcd(a, m)$ divides $b$. Assuming $\gcd(a, m)$ divides $b$, we have:
\[ t \cdot \gcd(a, m) = b \]

By multiplying Eq.3 by $t$, we get:
\[ a \cdot (t \cdot x_0) + m \cdot (t \cdot y_0) = t \cdot \gcd(a, m) \implies a \cdot X_0 + m \cdot Y_0 = b \]

Hence, given an equation to solve, we first verify if $\gcd(a, m)$ divides $b$. If so, a solution exists. Then, we find $x_0$ and $y_0$ using the Extended Euclidean Algorithm and multiply them by $t = \frac{b}{\gcd(a,m)}$ to obtain $X_0$ and $Y_0$.

Once $X_0$ and $Y_0$ are identified as solutions of Eq.2, we can substitute $x$ and $y$ as follows:
\[ x = X_0 + \frac{m}{\gcd(a,m)} \cdot n \]
\[ y = Y_0 + \frac{a}{\gcd(a,m)} \cdot n \]
where $n$ is an integer.

\section{Exploring Elliptic Curve Cryptography (ECC)}
\begin{itemize}
\item \textbf{Introduction:} While RSA offers a straightforward approach to cryptography with the Square and Multiply Algorithm, Elliptic Curve Cryptography (ECC) introduces a novel concept.
\item \textbf{Computations on Curves:} ECC operates on elliptic curves rather than integers, leading to the development of modern cryptographic techniques.
\item \textbf{Key Exchange:} ECC employs Elliptic Curve Diffie-Hellman (ECDH) for secure key exchange.
\item \textbf{Digital Signatures:} Signatures in ECC are generated using Elliptic Curve Digital Signature Algorithm (ECDSA).
\item \textbf{Security Benefits:} ECC's utilization of elliptic curves enables better security using smaller prime numbers.
\end{itemize}

Let's define two real numbers $a$ and $b$ such that:
\[ a, b \in \mathbb{R} \text{ and } 4a^3 + 27b^2 \neq 0 \]

Consider the curve:
\[ y^2 = x^3 + ax + b \]
where $(x, y) \in \mathbb{R}^2$. This curve is known as an Elliptic Curve.

\subsection{Elliptic Curve Mathematics}
\[ y^2 = x^3 + ax + b \]
\[ 4a^3 + 27b^2 \neq 0 \]

Let us consider two points $P(x_1, y_1)$ and $Q(x_2, y_2)$. We have three cases:
\begin{enumerate}
\item $x_1 \neq x_2, y_1 \neq y_2$
\item $x_1 = x_2, y_1 = -y_2$
\item $x_1 = x_2, y_1 = y_2$
\end{enumerate}

\subsection{Elliptic Curve Diffie-Hellman (ECDH)}
Let us consider a scenario where Alice and Bob want to exchange messages. They have a curve $E$ and a point $P$, and $(E, P)$ is public.

\section{Discrete Logarithm Problem}
The \textbf{Discrete Logarithm Problem (DLP)} is a foundational problem in cryptography that underpins the security of many algorithms, including the Diffie-Hellman key exchange and the Digital Signature Algorithm (DSA). It involves finding the exponent $x$ in the equation:
\[ g^x \equiv y \pmod{p} \]

where:
\begin{itemize}
\item $g$ is a generator of the group,
\item $p$ is a large prime number, and
\item $y$ is a known value.
\end{itemize}

\subsection{Why DLP is Hard}
The DLP is computationally hard because:
\begin{itemize}
\item There is no efficient algorithm for solving it in general cases within polynomial time.
\item It becomes infeasible as the prime $p$ grows larger (e.g., 2048-bit primes).
\end{itemize}

\subsection{Mathematical Properties}
\begin{enumerate}
\item $g^a \cdot g^b \equiv g^{a+b} \pmod{p}$ (Closure under multiplication)
\item $g^{-a} \cdot g^a \equiv 1 \pmod{p}$ (Existence of inverse)
\item Exponentiation in modular arithmetic is easy, but finding $x$ from $g^x$ is hard
\end{enumerate}

\section{Kerberos Authentication Protocol}
Kerberos is a secure, third-party authentication protocol designed for distributed networks. It provides mutual authentication between users and services by relying on a trusted third-party Key Distribution Center (KDC).

\subsection{Key Components}
\begin{enumerate}
\item \textbf{Key Distribution Center (KDC):} A trusted server that manages authentication by issuing tickets. It has two main components:
\begin{itemize}
\item \textbf{Authentication Server (AS):} Verifies users and issues a Ticket Granting Ticket (TGT).
\item \textbf{Ticket Granting Server (TGS):} Issues service-specific tickets based on the TGT.
\end{itemize}
\item \textbf{Principal:} A user or service that participates in Kerberos authentication.
\item \textbf{Ticket:} An encrypted data structure containing user credentials, session keys, and timestamps.
\end{enumerate}

\subsection{Authentication Workflow}
The Kerberos authentication process involves the following steps:
\begin{enumerate}
\item \textbf{Login and AS Request:}
\begin{itemize}
\item The user enters their credentials (username and password).
\item The client sends an authentication request to the AS, encrypted using the user's password-derived key.
\end{itemize}

\item \textbf{Ticket Granting Ticket (TGT):}
\begin{itemize}
\item The AS verifies the user's credentials and generates a TGT.
\item The TGT is encrypted using the KDC's secret key and sent to the client.
\end{itemize}

\item \textbf{Service Request to TGS:}
\begin{itemize}
\item The client presents the TGT to the TGS along with a service request.
\item The TGS validates the TGT and issues a service ticket.
\end{itemize}

\item \textbf{Accessing the Service:}
\begin{itemize}
\item The client sends the service ticket to the requested service.
\item The service validates the ticket and grants access.
\end{itemize}
\end{enumerate}

\subsection{Security Features}
\begin{itemize}
\item Mutual Authentication: Both the client and the service verify each other.
\item Replay Protection: Time-stamped tickets prevent reuse of old tickets.
\item Session Keys: Unique keys are generated for each session, ensuring secure communication.
\end{itemize}

\subsection{Advantages of Kerberos}
\begin{itemize}
\item Centralized authentication improves efficiency and security.
\item Mutual authentication ensures trust between users and services.
\item Time-stamped tickets mitigate replay attacks.
\end{itemize}

\subsection{Limitations of Kerberos}
\begin{itemize}
\item Relies on synchronized clocks; discrepancies can lead to authentication failure.
\item If the KDC is compromised, the entire system is at risk.
\item Initial password entry can expose credentials if not securely transmitted.
\end{itemize}

\subsection{Kerberos Ticket Structure}
A typical Kerberos ticket contains:
\begin{itemize}
\item Principal's identity (user or service)
\item Session key for secure communication
\item Ticket lifetime (start and expiry times)
\item Flags indicating ticket properties (e.g., renewable, forwardable)
\end{itemize}

\subsection{Kerberos Versions}
\begin{itemize}
\item \textbf{Kerberos v4:} Earlier version, used DES for encryption, now considered outdated.
\item \textbf{Kerberos v5:} Enhanced version, supports various encryption types, cross-realm authentication, and extended ticket properties.
\end{itemize}

\subsection{Applications of Kerberos}
\begin{itemize}
\item Secure login for distributed systems
\item Single Sign-On (SSO) implementation in corporate networks
\item Securing services such as email, file servers, and databases
\end{itemize}

\textbf{Kerberos ensures secure, efficient, and reliable authentication in distributed environments.}

\end{document}