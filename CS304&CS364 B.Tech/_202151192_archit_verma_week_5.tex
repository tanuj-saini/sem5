\documentclass[11pt]{article}
\usepackage[hmargin=1in,vmargin=1in]{geometry}
\usepackage{xcolor}
\usepackage{amsmath,amssymb,amsfonts,url,sectsty,framed,tcolorbox,framed}
\usepackage{algorithm,algpseudocode}


\newcommand{\pf}{{\bf Proof: }}
\newtheorem{theorem}{Theorem}
\newtheorem{lemma}{Lemma}
\newtheorem{proposition}{Proposition}
\newtheorem{definition}{Definition}
\newtheorem{remark}{Remark}
\newcommand{\qed}{\hfill \rule{2mm}{2mm}}

\begin{document}

%%%%%%%%%%%%%%%%%%%%%%%%%%%%%%%%%%%%%%%%%%%%%%%%%%%%%%%%%%%%%%%%%%%%%
\noindent
\rule{\textwidth}{1pt}
\begin{center}
    {\bf [CS304] Introduction to Cryptography and Network Security}
\end{center}
Course Instructor: Dr. Dibyendu Roy \hfill Winter 2023-2024\\
Scribed by: Archit Verma (202151192) \hfill Lecture (Week 5)
\\
\rule{\textwidth}{1pt}
%%%%%%%%%%%%%%%%%%%%%%%%%%%%%%%%%%%%%%%%%%%%%%%%%%%%%%%%%%%
%write here
% About Hill Cipher

\section*{AES (Advanced Encryption Standard)}
AES is a block cipher used for symmetric encryption.AES is widely used today as it is a much stronger than DES and triple DES despite being harder to implement.

\subsection*{AES-128}
\begin{itemize}
    \item \textbf{Block Size}: AES-128 operates on blocks of data, each consisting of 128 bits.
    \item \textbf{Key Size}: The secret key used in AES-128 encryption and decryption is 128 bits long.
    \item \textbf{Number of Rounds}: The encryption process in AES-128 consists of 10 rounds, each round applying specific transformations to the data.
    \item \textbf{Secret Key}: AES-128 employs a 128-bit secret key, which is used to perform both encryption and decryption operations.
\end{itemize}

\subsection*{Rounds}
Each round of AES encryption consists of the following operations:

\begin{enumerate}
    \item \textbf{SubBytes}: This operation substitutes each byte in the state array with another byte using an S-box. It provides non-linearity and confusion in the encryption process by transforming the input data.
    
    \item \textbf{ShiftRows}: In this operation, a cyclic left shift is performed on each row of the state array. The number of shifts is determined by the row index, ensuring diffusion of data within the block and improving resistance against cryptographic attacks.
    
    \item \textbf{MixColumns} (except for the last round): The MixColumns operation transforms each column of the state array using matrix multiplication, further contributing to diffusion and confusion. It spreads the input data across the output, enhancing the security of the encryption process. However, this operation is omitted in the last round to simplify the decryption process without compromising security.
\end{enumerate}


\subsubsection*{SubBytes}
\begin{itemize}
    \item The SubBytes transformation is a fundamental operation in AES, responsible for introducing non-linearity and confusion into the encryption process.
    \item It substitutes each byte in the state array with another byte using an S-box, which is a predefined lookup table.
    \item Transformation: $S: \{0,1\}^{8} \rightarrow \{0,1\}^{8}$.
    \item The input byte $x$ is represented as $x = X_0X_1 \ldots X_7$, where each $X_i$ represents a binary digit (0 or 1) corresponding to the individual bits of the byte.
    \item The output byte $y$ is computed using the S-box lookup table: $y = S(x)$.
    \item The S-box is constructed using a mathematical algorithm that involves finding the multiplicative inverse of polynomials modulo a fixed polynomial. This construction ensures good cryptographic properties and enhances the security of AES.
    \item The extended Euclidean algorithm is employed to find the multiplicative inverse, which is necessary for constructing the S-box and ensuring its cryptographic properties.
\end{itemize}

\subsubsection*{S-box}
The S-box substitution is defined by the following transformation:
\begin{align*}
    Y_0 &= S_1(X_0) \\
    Y_1 &= S_1(X_1) \\
    &\vdots \\
    Y_7 &= S_1(X_7)
\end{align*}
where $S_1$ is a non-linear function.

\subsubsection*{S-box Construction}
The S-box is constructed using a mathematical algorithm that involves finding the multiplicative inverses of polynomials modulo a fixed polynomial.


\subsubsection*{Extended Euclidean Algorithm}
To find the multiplicative inverse of a polynomial $p(x)$ under modulo $x^8 + x^4 + x^3 + x + 1$, we use the extended Euclidean algorithm:
\begin{align*}
    1 &= \text{gcd}(p(x), x^8 + x^4 + x^3 + x + 1) \\
    &= p(x) \cdot q(x) + h(x^8 + x^4 + x^3 + x + 1)
\end{align*}
where $q(x)$ is the quotient polynomial and $h(x)$ is the remainder polynomial.
\subsubsection*{Illustrating SubByte Algorithm}
To illustrate the SubBytes step, let's consider the input byte \( (01010011) \).

To represent the given input byte \( (01010011) \) as a polynomial, we assign each bit position a power of \( x \) in the polynomial representation. Starting from the left (most significant bit) to the right (least significant bit), we assign increasing powers of \( x \) to the bits where a 1 is present.

For the given input byte \( (01010011) \), the polynomial representation is derived as follows:

\[
(01010011) = 0 \cdot x^7 + 1 \cdot x^6 + 0 \cdot x^5 + 1 \cdot x^4 + 0 \cdot x^3 + 0 \cdot x^2 + 1 \cdot x^1 + 1 \cdot x^0 = x^6 + x^4 + x + 1
\]

\begin{itemize}
    \item We represent the input byte as a polynomial: \( x^6 + x^4 + x + 1 \). This representation is derived from the bit positions where 1s are present.
    \item We have an S-box transformation, \( S: \{0,1\}^8 \rightarrow \{0,1\}^8 \), which substitutes each byte using a lookup table.
    \item We perform the substitution: \( S(01010011) = (11001010) \).
\end{itemize}

Now, let's compute \( b_7b_6 \ldots b_0 \):
\begin{align*}
    b_0 &= (a_0 + a_4 + a_5 + a_6 + a_7 + c_0) \mod 2 \\
    &= (0 + 0 + 0 + 1 + 1 + 1) \mod 2 \\
    &= 1 \\
    b_1 &= 0 \\
    &\vdots \\
    b_7 &= 1
\end{align*}

Thus, \( b_7b_6b_5 \ldots b_0 = 11101101 \).

Finally, we find the output byte from the lookup table:
\[ \text{subbyte}(53) = \text{ED} \]

To improve efficiency, we can store the calculations beforehand in a lookup table.



\subsubsection*{ShiftRows}
\begin{itemize}
    \item The ShiftRows operation performs a cyclic left shift on each row of the state array, ensuring diffusion of data within the block.
    \item It is essential for providing diffusion and improving the resistance against cryptographic attacks.
    \item Example:
    \[
    \begin{bmatrix}
    b_0 & b_1 & b_2 & b_3 \\
    b_4 & b_5 & b_6 & b_7 \\
    b_8 & b_9 & b_{10} & b_{11} \\
    b_{12} & b_{13} & b_{14} & b_{15}
    \end{bmatrix}
    \rightarrow
    \begin{bmatrix}
    b_0 & b_1 & b_2 & b_3 \\
    b_5 & b_6 & b_7 & b_4 \\
    b_{10} & b_{11} & b_8 & b_9 \\
    b_{15} & b_{12} & b_{13} & b_{14}
    \end{bmatrix}
    \]
\end{itemize}

\subsubsection*{MixColumns}
\begin{itemize}
    \item The MixColumns operation transforms each column of the state array using matrix multiplication, further contributing to diffusion and confusion.
    \item It involves multiplying each column with a fixed matrix to ensure the spread of input data across the output, enhancing the security of the encryption process.
    \item Example:
    \[
    \begin{bmatrix}
    b_0 & b_1 & b_2 & b_3 \\
    b_4 & b_5 & b_6 & b_7 \\
    b_8 & b_9 & b_{10} & b_{11} \\
    b_{12} & b_{13} & b_{14} & b_{15}
    \end{bmatrix}
    \rightarrow
    \begin{bmatrix}
    b'_0 & b'_1 & b'_2 & b'_3 \\
    b'_4 & b'_5 & b'_6 & b'_7 \\
    b'_8 & b'_9 & b'_{10} & b'_{11} \\
    b'_{12} & b'_{13} & b'_{14} & b'_{15}
    \end{bmatrix}
    \]
\end{itemize}

\subsection*{Last Round}
\begin{itemize}
    \item The last round differs from the others as it does not include the MixColumns operation, simplifying the decryption process without compromising security. This omission reduces the complexity of decryption algorithms while maintaining the same level of security.
\end{itemize}

\subsection*{Key Schedule}
\begin{itemize}
    \item The Key Schedule generates round keys from the main secret key to be used in each round of the AES encryption process.
    \item It involves applying a series of transformations to the main secret key to derive subkeys for each round, ensuring the security of the encryption process.
\end{itemize}

\subsection*{Key Schedule Algorithm (KSA)}
The Key Schedule Algorithm is responsible for generating round keys from the main secret key.

\subsubsection*{Left Circular Shift}
The function \texttt{ROTWORD(B0,B1,B2,B3)} performs a left circular shift on the input bytes, resulting in \((B1,B2,B3,B0)\).

\subsubsection*{Size of \(B_i\)}
Each \(B_i\) is 8 bits.

\subsubsection*{Subword Transformation}
The \texttt{subword} transformation converts each byte \(B_i\) to \(B_i'\) using the SubBytes operation.

\subsubsection*{Constants for Round Keys}
There are 10 constants used for the 11 round keys: \(Rcon[1] = 01000000\), \(Rcon[2] = 36000000\), and so on.

\subsubsection*{Key Expansion Algorithm}
The key expansion algorithm computes the round keys from the main secret key. It iterates through 44 words (each word being 32 bits or 4 bytes) to generate the round keys.
\begin{algorithm}
\caption{Key Expansion Algorithm}
\begin{algorithmic}[1]
\State $w[0] \ldots w[3] =$ Key
\For{$i = 4$ to $43$}
    \State temp $\gets w[i-1]$
    \If{$i \mod 4 = 0$}
        \State temp $\gets$ \texttt{subword(ROTWORD(temp))} $\oplus$ \texttt{Rcon[i/4]}
    \EndIf
    \State $w[i] \gets w[i-4] \oplus$ temp
\EndFor
\State \Return $w[0] \ldots w[43]$
\end{algorithmic}
\end{algorithm}

The resulting round keys are \(w[0] \ldots w[43]\), where each key is 128 bits.

\subsection*{Modes of Operation}
\begin{itemize}
    \item ECB (Electronic Codebook)
    \item CBC (Cipher Block Chaining)
    \item CFB (Cipher Feedback)
    \item OFB (Output Feedback)
    \item IGE (Infinite Garble Extension)
\end{itemize}

\subsubsection*{ECB (Electronic Codebook)}
ECB mode encrypts each block of plaintext separately using the same key.

\subsubsection*{Encryption in ECB}
The encryption process in ECB mode involves dividing the plaintext into blocks, applying the encryption algorithm to each block using the secret key, and concatenating the resulting ciphertext blocks.

\subsubsection*{Decryption in ECB}
The decryption process in ECB mode is the inverse of encryption, where each ciphertext block is decrypted using the same secret key.

\subsubsection*{Mathematical Representation}
Let \(P\) be the plaintext, \(C\) be the ciphertext, and \(K\) be the secret key. Then, the encryption and decryption functions can be represented as:
\[ C_i = \text{AES-128-Encrypt}(P_i, K) \]
\[ P_i = \text{AES-128-Decrypt}(C_i, K) \]

\subsubsection*{CBC (Cipher Block Chaining)}
CBC mode encrypts each plaintext block by XORing it with the previous ciphertext block before encryption.

\subsubsection*{Encryption in CBC}
The encryption process in CBC mode involves XORing each plaintext block with the previous ciphertext block (or the initialization vector for the first block), encrypting the result using the secret key, and then updating the ciphertext block for the next iteration.

\subsubsection*{Decryption in CBC}
The decryption process in CBC mode is similar to encryption, but it involves decrypting each ciphertext block and XORing it with the previous ciphertext block (or the initialization vector for the first block) to obtain the plaintext.

\subsubsection*{Mathematical Representation}
Let \(IV\) be the initialization vector. Then, the encryption and decryption functions in CBC mode can be represented as:
\[ C_i = \text{AES-128-Encrypt}(P_i \oplus C_{i-1}, K) \]
\[ P_i = \text{AES-128-Decrypt}(C_i, K) \oplus C_{i-1} \]


\section*{Stream Ciphers}

Stream ciphers are a type of symmetric encryption algorithm that encrypts plaintext data one bit or byte at a time, typically in a continuous stream. They are often used in situations where the plaintext data is transmitted continuously, such as in real-time communication systems like voice or video calls, or in applications where low latency is critical.

\section*{Overview}

Let the message be a sequence of bits:
\[ 01010100 \, 01111000 \]

Let
\[ 10101000 \, 01111000 \]
be a sequence of random bits.

The ciphertext is obtained by performing bitwise XOR operation between the message and the random sequence:
\[ 01010100 \oplus 10101000 = 11111100 \]
\[ 01111000 \oplus 01111000 = 00000000 \]

So, the ciphertext is:
\[ 11111100 \, 00000000 \]

where
\[ 10101000 \, 01111000 \]
is the random sequence.

Perfect Secrecy:
Given a ciphertext, the message can be any binary string of length 8.

Impracticalities:
\begin{enumerate}
    \item Difficult to get random bits.
    \item Sequence must be pre-distributed to both sender and receiver.
    \item Requires a random sequence as long as the message.
\end{enumerate}

\section*{Additive Stream Ciphers}

Additive stream ciphers use a pseudorandom generator (PRG) to produce the pseudorandom sequence. The PRG extends a short "seed" into a long "pseudorandom" string, i.e.,
\[ \text{PRG}(seed) = \text{pseudorandom sequence} \]

The seed is the secret key. The security of the additive stream cipher depends on the design of the PRG.

\section*{Security of PRG}

PRG must be unpredictable for cryptographic use. This includes:
\begin{enumerate}
    \item Next bit test: Given an initial segment, it should not be possible to efficiently guess the next bit.
    \item Statistical Tests: The generated pseudorandom sequence should pass all polynomial time statistical tests.
\end{enumerate}

The above notions are equivalent for stream cipher use.
\end{document}
