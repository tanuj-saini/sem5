\documentclass[11pt]{article}
\usepackage[hmargin=1in,vmargin=1in]{geometry}
\usepackage{xcolor}
\usepackage{amsmath,amssymb,amsfonts,url,sectsty,framed,tcolorbox,framed}
\newcommand{\pf}{{\bf Proof: }}
\newtheorem{theorem}{Theorem}
\newtheorem{lemma}{Lemma}
\newtheorem{proposition}{Proposition}
\newtheorem{definition}{Definition}
\newtheorem{remark}{Remark}
\newcommand{\qed}{\hfill \rule{2mm}{2mm}}

\begin{document}
%%%%%%%%%%%%%%%%%%%%%%%%%%%%%%%%%%%%%%%%%%%%%%%%%%%%%%%%%%%%%%%%%%%%%
\noindent
\rule{\textwidth}{1pt}
\begin{center}
    {\bf [CS304] Introduction to Cryptography and Network Security}
\end{center}
Course Instructor: Dr. Dibyendu Roy \hfill Winter 2023-2024\\
Scribed by: Archit Verma (202151192) \hfill Lecture (Week 1)
\\
\rule{\textwidth}{1pt}
%%%%%%%%%%%%%%%%%%%%%%%%%%%%%%%%%%%%%%%%%%%%%%%%%%%%%%%%%%%
%write here
\section*{Introduction}
Cryptography, the practice of safeguarding communication and information through codes and ciphers, is pivotal for ensuring the confidentiality, integrity, and authenticity of data in various applications.

\section*{Cryptography}
\begin{itemize}
    \item Cryptography involves devising codes and algorithms to encrypt information, rendering it unreadable to unauthorized parties.
    \item The primary objective of cryptography is to facilitate secure communication and shield data from unauthorized access or tampering.
\end{itemize}

\subsection*{Cryptanalysis}
\begin{itemize}
    \item Cryptanalysis, the science of analyzing and breaking codes and ciphers, aims to decipher encrypted messages without the proper key.
    \item The ongoing interplay between cryptography and cryptanalysis propels the evolution of secure communication methods.
\end{itemize}

\subsection*{Symmetric Cryptography}
\begin{itemize}
    \item Symmetric cryptography, also known as secret-key cryptography, employs the same key for both encryption and decryption.
\end{itemize}

\subsection*{Asymmetric Cryptography}
\begin{itemize}
    \item Asymmetric cryptography, or public-key cryptography, utilizes a pair of keys: public and private keys.
    \item Information encrypted with the public key can only be decrypted with the corresponding private key, and vice versa.
\end{itemize}

\section*{Cryptographic Functions}
Cryptography extends beyond mere encoding and decoding; it encompasses crucial functions ensuring the security and reliability of communication. Key operations integral to cryptographic systems include:

\subsection*{Confidentiality}
Confidentiality ensures sensitive information remains private and accessible only to authorized parties. Cryptographic techniques, such as encryption, transform data into an unreadable format for those without the appropriate decryption key.

\subsection*{Integrity}
Integrity guarantees data remains unaltered and trustworthy during transmission or storage. In cryptography, integrity is maintained through mechanisms like hash functions.

\subsection*{Authentication}
Authentication verifies the identity of communicating parties and ensures the message's origin. Cryptographic functions, like digital signatures, play a vital role in authentication by confirming the sender's identity and validating the message's integrity, preventing unauthorized entities from masquerading as legitimate sources.

\subsection*{Non-repudiation}
Non-repudiation prevents individuals from denying their involvement in a communication or transaction. Digital signatures provide evidence of the message's origin and authenticity, strengthening accountability in digital interactions.

\section*{Different Types of Ciphers}

\subsection*{Caesar Cipher}
The Caesar Cipher is a substitution cipher shifting each letter in the plaintext by a fixed number of positions using modular arithmetic.

\subsection*{Transposition Cipher}
Transposition ciphers rearrange characters in the plaintext without altering their identities through permutation.

\subsection*{Substitution Cipher}
Substitution ciphers replace each plaintext character with another according to a predefined key.

\subsection*{Plaintext Cipher}
Plaintext ciphers introduce complexity or obfuscation without altering the message's content.

\subsection*{Affine Cipher}
The Affine Cipher combines linear and modular arithmetic, using two functions for multiplication and addition, providing a more complex substitution method than the Caesar Cipher.

\subsection*{Extended Euclidean Algorithm}
The Extended Euclidean Algorithm, used in the Affine Cipher, finds the modular inverse of a number, crucial for decryption in asymmetric encryption systems.

\subsection*{Playfair Cipher}
The Playfair Cipher encrypts digraphs based on their positions in a key table, typically a 5x5 grid of letters. I and J are treated the same, and the encryption process involves rules for handling pairs of letters within the same row, column, or rectangle.

\end{document}